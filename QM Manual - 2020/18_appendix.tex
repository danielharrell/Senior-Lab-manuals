\chapter{Appendix: ESU, EMU, and SI}

\chapterprecis{Units and Systems of Measurement}

\makeoddhead{myheadings}{\emph{Appendix: Electricity and Magnetism }}{}{\thepage}
\makeevenhead{myheadings}{\thepage}{}{\emph{Systems of Measurement and Lorentz Force}}

\renewcommand{\theequation}{\arabic{equation}}


\section*{Measurement of Electrical and Magnetic Quantities}

Before the adoption in 1960 of the International System of Units (usually 
abbreviated as SI, for \emph{Système International}),
scientists used two main systems of units for electric and magnetic
quantities, depending on whether they were investigating primarily
electromagnetic or electrostatic phenomena. For us, whose authors use these older
units and whose measuring instruments use modern units, there is a practical 
necessity to be able to convert between them. But the differences and relations 
among these schemes of measurement also depend on the nature of electricity
and reveal something about the difficulties inherent to the project of measurement.

The ESU (electrostatic) and EMU (electromagnetic) systems of units
both define measures for charge, current, and electric potential. (The SI system, perhaps
counter-intuitively, makes the measure of current fundamental.) In both systems, as well 
as in the International System, these units are named after early investigators of 
electrical phenomena: Coulomb, Ampere, and Volta. For practical purposes, all that is 
required is to know the ratios among the various systems of units, as given in the 
following table.

\begin{center}
\begin{tabular}{l l l}
\textbf{SI units} & \textbf{EMU} & \textbf{ESU}\\
$1$ coulomb & $0.1$ abcoulomb & $3\!\times\!10^9$ statcoulombs\\
$1$ ampere & $0.1$ abampere & $3\!\times\!10^9$ statamperes\\
$1$ volt & $10^8$ abvolts & $1/300$ statvolt\\
\end{tabular}
\end{center}

The dimensions of the units depend on which phenomenon is taken as definitive: 
the mutual attraction or repulsion of two charged bodies, or the magnetic field 
produced by charged bodies in motion. The former gives rise to the ESU system
and the latter to the EMU system. The sizes of the units depend on the size of
what are taken as the basic units of length, mass, and time. For ESU and EMU,
those are the centimeter, the gram, and the second. The International System
of Units differs in the size of its fundamental units for these dimensions, as
it uses the meter, the kilogram, and the second. (Accordingly, these or related
systems are sometimes referred to as CGS and MKS.) The derived SI units for, say, 
force and work, are defined with reference to the fundamental units: one newton 
is defined as the amount of force necessary to accelerate one kilogram of mass 
at the rate of one meter per second per second, one joule of work is done when a 
force of one newton acts on a body in the direction of its motion over a distance 
of one meter, and so forth.

As noted above, the International System makes one electrical quantity a ``base unit,'' 
namely, the unit of current, the ampere. It is a base unit in the sense that the other
electrical quantities (e.g., charge, voltage, capacitance) are defined in terms of it. The ampere
has been defined from 1946 until the present in terms of the effect discovered by 
Oersted, as "the constant current which, 
if maintained in two straight parallel conductors of infinite length, of 
negligible circular cross-section, and placed 1 meter apart in vacuum, would produce 
between these conductors a force equal to $2\!\times\!10^{-7}$ newtons per meter 
of length." At the time of this writing, the 26th General Conference of Weights
and Measures is scheduled to vote (on Nov. 16, 2018) to redefine the ampere
in terms of the elementary electrical charge, that carried by an electron (or proton).
Thenceforward, the ampere will be defined 
\begin{quote}
by taking the fixed numerical value of the elementary charge $e$ to be $1.602\, 176\, 634\, 
\times \, 10^{-19}$ when expressed in the unit C, which is equal to A s.
\end{quote}
The ampere, then, still nominally a base unit, is to be defined in terms of the coulomb, 
which is a quantity of charge specified by a definite \emph{number} of elementary charges.
The means by which this elementary charge is measured are unlike those of the
electrostatic and electromagnetic systems of old. It relies on the physics of semiconductors,
and the counter-intuitive quantum-mechanical behavior of individual electrons. A good source
for further information is the Physical Measurement Laboratory, a major operating unit of the 
National Institute for Standards and Technology which develops and disseminates national
standards for measurement of physical quantities.


\subsection*{Electric Field Intensity and Electric Potential}

Whichever system they use, our authors depend on being able to speak
quantitatively about not only charge and current, but also \emph{electric field
intensity} and \emph{electric potential}. A few notes about the dimension
and measure of those quantities follow.

\emph{Electric intensity} or \emph{electric field strength} is expressed
in terms of force per unit charge. The nameless unit is easily defined in the
ESU and EMU systems: a field has strength 1 at a point if a charge of one
statcoulomb (ESU) or 1 abcoulomb (EMU) placed there would experience a force 
of one dyne. Generally, within one of these systems of measurement, a field of 
strength $E$ will exert a force $F$ on a charge $q$ numerically equal to $F = Eq$.

An especially convenient way of speaking about the intensity and extent of
an electric field at once is as \emph{electric potential} or \emph{electric 
potential difference}, otherwise known as \emph{voltage}. Except in the imaginary scenario
necessary for a clear definition, a charge is never merely ``placed'' in an electric
field; an actual charge feels a force, and responds by being accelerated. If we set up an
electric field in a region of space---say, by charging two parallel plates
with opposite charges---it can be useful to think about the quantitative
measure of the electrical situation in that region in terms of how much kinetic energy a 
charged body would acquire in moving from one to another point within it. The electric potential
difference between two locations in an electric field is equal to the work that
would be done on a unit charge in moving from the one spot to the other. Potential
is measured as work per charge. A feature of a ``conservative'' field of force 
(of which both the electric and the gravitational field are examples) is that
the same difference in kinetic energy will result \emph{no matter which path the
body takes to get from the one point to the other}. Even if a charged body is compelled
to take another path than the one it would if only subject to the electric force, the energy 
it gains due to the action of the field in going from one point to another (or, if opposed
by the force, the work required to get it from the one point to the other),
is uniquely specified by the potential difference between the two points multiplied by the 
quantity of the charge on the body.
This \emph{path-independence} is characteristic of conservative fields of force and
greatly simplifies many calculations.

In the ESU, EMU, and SI systems alike, the unit of electric potential difference is
defined as the work done on or energy gained by one unit of charge in passing from
one place to another in the field. In SI units, it takes 1 joule of work to move
a charge of 1 coulomb across a potential difference of 1 volt. 

Given that the intensity of the electric field $E$ is measured as force per charge
$(f/q)$, and work $W$ is measured as force $f$ acting over a distance $s$, we can
see that for a uniform electric field, the potential difference $(V=W/q)$ between 
two points in it is simply the field strength $E$ times the distance $s$.
\begin{equation*}
E \cdot s = \frac{f}{q}\cdot s = \frac{f\cdot s}{q} = \frac{W}{q} = V.
\end{equation*}
In SI units, then, the dimension of electric field strength can also be expressed
as \emph{volts per meter}. In practical applications, it is often convenient to
think of it in this way, and thus be able to calculate the strength of a uniform field given
the voltage and the distance.
\begin{equation*}
E = \frac{V}{s}.
\end{equation*}
